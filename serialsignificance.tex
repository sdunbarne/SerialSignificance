
\documentclass[12pt]{article}

\input{../../../../etc/macros} %\input{../../../../etc/mzlatex_macros}
\input{../../../../etc/pdf_macros}

\bibliographystyle{plain}

\begin{document}

\myheader \mytitle

\hr

\sectiontitle{Simple Serial Significance Testing}

\hr

\usefirefox

\hr

% \visual{Study Tip}{../../../../CommonInformation/Lessons/studytip.png}
% \section*{Study Tip}

% \hr

\visual{Rating}{../../../../CommonInformation/Lessons/rating.png}
\section*{Rating} %one of
% Everyone: contains no mathematics.
% Student: contains scenes of mild algebra or calculus that may require guidance.
% Mathematically Mature: may contain mathematics beyond calculus with proofs.
Mathematicians Only:  prolonged scenes of intense rigor.

\hr

\visual{Section Starter Question}{../../../../CommonInformation/Lessons/question_mark.png}
\section*{Section Starter Question}

Given a statistical test for the hypothesis that a state from a
Markov chain is ``unusual'' with respect to some function
labeling the states, what would a \emph{Type II error} for the statistical
test mean?  

Given a statistical test for the hypothesis that a state from a
reversible Markov chain is ``unusual'' with respect to some function
labeling the states, what would a \emph{Type I error} for the statistical
test mean?

\hr

\visual{Key Concepts}{../../../../CommonInformation/Lessons/keyconcepts.png}
\section*{Key Concepts}

\begin{enumerate}
    \item
        The \( \sqrt{\epsilon} \) test is:  Observe a trajectory \( X_0,
        X_1, X_2, \dots, X_n \) from the state \( X_0 \) for any fixed \(
        n \). The event that \( v(X_0) \) is an \( \epsilon \)-outlier
        among \( v(X_0), \dots v(X_n ) \) is significant at \( p = \sqrt
        {2\epsilon} \) under the null hypothesis that \( X_0 \sim \pi \).
    \item
    \item
\end{enumerate}

\hr

\visual{Vocabulary}{../../../../CommonInformation/Lessons/vocabulary.png}
\section*{Vocabulary}
\begin{enumerate}
    \item
        A \defn{value function} or \emph{label} is a function \( v :
        \mathcal{X} \to \Reals \).  The value function or labels have
        auxiliary information about the states of the chain and are not
        assumed to have any relationship to the transition probabilities
        of the Markov chain.
    \item
        A real number \( \alpha_0 \) is an \defn{\( \epsilon \)-outlier}
        among \( \alpha_0, \alpha_1, \dots \alpha_n \) if there are, at
        most, \( \epsilon(n + 1) \) indices \( i \) for which \( \alpha_i
        \le \alpha_0 \), in other words \( \card{ \setof{i}{\alpha_i <
        \alpha_0}} \le \epsilon(n+1) \).
    \item
        Say \( \alpha_j \) is \emph{among-\( \ell \)-smallest} among \(
        \alpha_0, \dots \alpha_n \) if there are at most \( \ell \)
        indices \( i \ne j \) from \( 0, \dots, n \) such that the label
        of \( \alpha_i \) is at most the label of \( \alpha_j \).  That
        is, \( \alpha_j \) is among-\( \ell \)-smallest among \( \alpha_0,
        \dots \alpha_n \) if \( \card{ \setof{i}{0 \le i \le n, v(\alpha_i)
        < v(\alpha_j)}} \le \ell \).  In particular, \( \alpha_j \) is
        among-\( 0 \)-smallest from \( \alpha_0, \dots \alpha_n \) when
        its label is the unique minimum label.
\end{enumerate}

\hr

\section*{Notation}
\begin{enumerate}
    \item
        \( X \), \( X_0 \), \( X_n \) -- Markov chain, starting state,
        general step of the chain.
  \item \( \mathcal{X} \) -- State space of a Markov chain
  \item \( v :  \mathcal{X} \to \Reals \) -- value or label function
  \item \( \pi \) -- stationary distribution
  \item \( x_0 \) -- a given state from the state space
  \item \( \alpha_0, \alpha_1, \dots \alpha_n \) -- arbitrary real
    numbers for the definition of outliers
  \item \( \epsilon \) -- small real number
  \item \[
              \rho(j; \ell, n) = \Prob{X_j \text{ is
                  among-$\ell$-smallest from } X_0, \dots, X_n }
            \]
  \item   \( \rho(j; 0, n \given \sigma)
            \) -- the probability that \( X_j = \sigma \) has the unique
            minimum value in the Markov chain of length \( n \) given
            that the Markov passes through \( \sigma \) at step \( j
            \).
  \item \( p_{\nu}^n \) -- the
probability that in a \( n \)-step stationary trajectory \( X_0, X_1,
\dots X_n \), the minimum \( v \) label value occurs at \( X_i \).
\item \( N \) -- a large integer
\item \( \zeta_{\nu} = \pm 1 \)  --  Bernoulli random variable with
equal probability \( \frac{1}{2} \)
\item   \( \tau_2 \) -- the relaxation time
  for \( X \) defined as \( 1/ (1-\lambda_2) \) 
\item  \( \lambda_2 \) --  the spectral gap for \( X \)
\end{enumerate}

\visual{Mathematical Ideas}{../../../../CommonInformation/Lessons/mathematicalideas.png}
\section*{Mathematical Ideas}

% To Do:
% \begin{enumerate}
%     \item
%         Find a more descriptive name for the outliers.
%     \item
%         Reorganize proofs and theorems to a logical order, maybe use
%         hand drawn graph, also reorganize Theorem references.
% \end{enumerate}

Following the work of Chikina, Frieze, Mattingly and Pegden
\cite{doi:10.1080/2330443X.2020.1806763, Chikina2860, Chikina2019} this
section derives statistical tests to detect that a given state of a
reversible Markov chain was \emph{not} chosen from a stationary
distribution.  In particular, given a value function for the states of
the Markov chain, the goal is to show rigorously the given state is an
outlier with respect to the values, by establishing a \( p \) value
under the null hypothesis the given state comes from a stationary
distribution of the chain.

In simplest form, the test is the following:  Given a particular state
in the Markov chain, take a random walk from the state for any
number of steps.  The simplest statistical test is that observing that
the presented state is an \( \epsilon \)-outlier on the walk is
significant at \( p = \sqrt{2\epsilon} \) under the null hypothesis that
the state was chosen from a stationary distribution.  The test assumes
nothing about the Markov chain beyond reversibility.  With the simple
test, significance at \( p \approx \sqrt{\epsilon} \) is the best
possible in general.  Compound tests using multiple random
walks can provide better significance at \( p \approx \epsilon \).

\subsection*{Theoretical Background}

The essential problem in statistics is to bound the probability of a
surprising observation under a null hypothesis that observations are
drawn from some probability distribution.  Often the bounding
calculation is difficult.  First, defining the way in which the outcome
is surprising requires care.  Second, the correct choice of the unbiased
distribution implied by the null hypothesis may not be immediately
clear.
Third, the \( p \) value calculations may be difficult when
the observation is surprising in a simple way and the null hypothesis
distribution is known but where there is no simple algorithm to draw
samples from this distribution.  In the third case, the best candidate
method to sample from the null hypothesis is often through a Markov
chain, which essentially takes a long random walk on the possible values
of the distribution.  Under suitable conditions, theorems about Markov
chains guarantee that the chain converges to its stationary
distribution, sometimes at a known rate, allowing a random sample to be
drawn from a distribution quantifiably close to the target distribution.
This principle is behind Markov chain Monte Carlo sampling and
statistical physics models.

A problem in applications of Markov chains is the usually unknown rate
at which the chain converges to the stationary distribution.  It is
rare to have rigorous results on the mixing time of a real-world Markov
chain, which means that, in practice, sampling proceeds by running a
Markov chain for a ``long time'' and hoping that sufficient mixing has
occurred.

In this section, the problem is assessing statistical significance in a
Markov chain \emph{without requiring results on the mixing time} of the
chain or indeed, any special structure at all in the chain beyond
reversibility.  Consider a reversible Markov chain \( X \) on a state
space \( \mathcal{X} \) with an associated label or value function \( v
: \mathcal{X} \to \Reals \)  (The notation here is \( v \) for the
\emph{value} or label function instead of \( \omega \) as in
\cite{Chikina2860}, to avoid overloading notation.  Elsewhere \( \omega \)
represents sample points in the sample space.) The labels or values
give auxiliary information without assuming any relationship to the
transition probabilities of \( X \).  The goal is to show that a given
state \( x_0 \) is unusual for states drawn from a stationary
distribution \( \pi \).  With good bounds on the mixing time of \( X \),
sampling from a distribution of \( v(\pi) \) gives a rigorous \( p \)
value for the significance of the smallness of the label of \( x_0 \).
However, such convergence bounds are rarely available.

Proposition~%
\ref{thm:significance:basethm} detects that \( X_0 = x_0 \) is unusual
relative to states chosen randomly according to \( \pi \).  The test
does not require bounds on the mixing rate of \( X \).  First, a more
precise definition of ``unusual relative to states chosen randomly''.

\begin{definition}
    A \defn{value function},~%
    \index{value function}
    also called a \emph{label}, is a function \( v :  \mathcal{X} \to
    \Reals \).  The value function has auxiliary information
    about the states of the chain and has no
    relationship to the transition probabilities of the Markov chain.
\end{definition}

\begin{remark}
    For the statistical significance tests in this section the only
    relevant feature of the value function is the ranking it imposes on
    the elements of \( \mathcal{X} \).
\end{remark}

\begin{definition}
    A real number \( \alpha_0 \) is an \defn{\( \epsilon \)-outlier}%
    \index{\( \epsilon \)-outlier}
    among \( \alpha_0, \alpha_1, \dots \alpha_n \) (with repetitions
    possible) if there are, at most, \( \epsilon(n + 1) \) indices \( i \)
    for which \( \alpha_i \le \alpha_0 \), in other words \( \card{%
    \setof{i}{\alpha_i \le \alpha_0}} \le \epsilon(n+1) \).
    Alternatively, \( \alpha_0 \) is in the smallest \( \epsilon \)-fraction
    of the \( n \) ranked values of \( \alpha_0, \alpha_1, \dots \alpha_n \).
  \end{definition}

  \begin{remark}
    For \( \alpha_0 \) to be an \( \epsilon \)-outlier says that \(
    a\pha \) is relatively more rare than the \( \epsilon
    \)-proportion that might be expected if the indices were uniformly
    distributed the range of values.     For example, consider the set of \( 11 \) values
    \( \set{\alpha_0 = 3, \alpha_1 = 1, \alpha_2 = 4, \alpha_3 = 1,
      \alpha_4 = 5, \alpha_5 = 9, \alpha_6 = 2, \alpha_7 = 6, \alpha_8
      = 5, \alpha_9 = 3, \alpha_{10} = 5} \).  Then \(
    \alpha_0 = 1 \) is \emph{not} an \( 0.1 \)-outlier since \( 2 =
    \card{%
    \setof{i}{\alpha_i \le \alpha_0}} \ge \epsilon(n+1) = 1.1 \).  In
  fact,  \( \alpha_0 = 1 \) \emph{is not} a \( 0.2 \)-outlier (the
  cardinality is \( 3 \), greater than \(2.2\))
  or even a \( 0.3 \)-outlier (the cardinality is \( 5 \), greater
  than \( 3.3 \).   This example says \( \alpha_0 = 1 \) is not
  rare among the small fractions of the indices.

  \end{remark}

The following definition uses counting instead of proportions to express
the smallness of a real number relative to a set.  The proof of 
Proposition~\ref{thm:significance:basethm}  uses this definition.

\begin{definition}
    Say \( \alpha_j \) is \defn{among-\( \ell \)-smallest}%
    \index{among-\( \ell \)-smallest}
    among \( \alpha_0, \dots \alpha_n \) if there are at most \( \ell \)
    indices \( i \ne j \) from \( 0, \dots, n \) such that  \(
    \alpha_i \) is at most as large as \( \alpha_j \).  That is, \(
    \alpha_j \) is among-\( \ell \)-smallest among \( \alpha_0, \dots
    \alpha_n \) if
    \[
      \card{ \setof{i}{i \ne j, v(\alpha_i) \le v(\alpha_j)}}
      \le \ell.
    \]  In particular, \( \alpha_j \) is among-\( 0 \)-smallest
    from \( \alpha_0, \dots \alpha_n \) when its label is the unique
    minimum label.
\end{definition}

\begin{remark}
    The definitions are connected:
    \begin{itemize}
        \item
            If \( \alpha_0 \) is an \( \epsilon \)-outlier, then \(
            \alpha_0 \) is among-\( \lfloor \epsilon (n+1) \rfloor \)-smallest.
        \item
            If \( \alpha_0 \) is among-\( \ell \)-smallest, then \(
            \alpha_0 \) is an \( \frac{\ell + 1}{n+1} \)-outlier.
    \end{itemize}
\end{remark}

\begin{example}
    Consider the set of \( 11 \) real numbers drawn at random uniformly
    from \( [0,1] \) and presented in sorted order:
    \[
        0.006, 0.042, 0.359, 0.381, 0.399, 0.523, 0.533, 0.702, 0.762, 0.921,
        0.963
      \] so \( n = 10 \) and let \( \epsilon = 0.1 \).

    Then \( 0.006 \)
    is an \( 0.1 \)-outlier since the number of indices for which \(
    \alpha_i \le 0.006 \), specifically \( 1 \), is less than \(
    \epsilon(n+1) = 0.1\cdot (10 +1) = 1.1 \).
    See Figure~\ref{fig:serialsignificance:epsoutlier} for an
    illustration of this example.
    However, \( 0.006 \)
    is not an \( 0.05 \)-outlier since the number of indices for which \(
    \alpha_i \le 0.006 \), specifically \( 1 \), is more than \(
    \epsilon(n+1) = 0.05 \cdot (10 +1) = 0.55 \).  As a further
    example, there are \( 2 \) indices for which \( \alpha_i \le 0.042
    \) so \( 0.042 \) is not an \( 0.05 \), \( 0.1 \) or
    \(0.15 \)-outlier, but it is a \( 0.2 \)-outlier 

    \begin{figure}
      \centering
      \begin{asy}
size(5inches);

real myfontsize = 12;
real mylineskip = 1.2*myfontsize;
pen mypen = fontsize(myfontsize, mylineskip);
defaultpen(mypen);

real eps = 0.1;
pair[] alpha = {(0.006,0), (0.042,0), (0.359,0), (0.381,0), (0.399,0),
			 (0.523,0), (0.533,0), (0.702,0), (0.762,0),
			 (0.921,0), (0.963,0)};
pair h = (0,0.05);

draw((0,0)--(1,0));
for(int i=0; i<11; ++i) {
  dot(alpha[i]);
}
draw(circle(alpha[0], 0.01), red);

draw( (alpha[0]+h)--(alpha[10]+h), Bars);
// draw( (alpha[0]+h)--(eps * (alpha[10]-alpha[0]) + h), Bars);

// label("$\epsilon$", ((1/2)* eps * (alpha[10]-alpha[0])) + 1.5*h);
label("$\alpha_0$", alpha[0]-h);
label("$\alpha_n$", alpha[10]-h);
      \end{asy}
      \caption{Example of $\epsilon$-outliers with $11$ random values
        in $[0,1]$ and $\epsilon = 0.1$.}
      \label{fig:serialsignificance:epsoutlier}
    \end{figure}

    Also, \( 0.006 \) is among-\( 0 \)-smallest since there are \( 0 \)
    indices \( i \ne j \) from \( 0, \dots, n \) such that the value of \(
    \alpha_i \) is at most the value of \( 0.006 \).  Additionally, \(
    0.042 \) is among-\( 1 \)-smallest since there is \( 1 \) index \( i
    \ne j \) from \( 0, \dots, 10 \) such that the value of \( \alpha_i \)
    is at most the value of \( 0.042 \), specifically the value \( 0.006
    \).

    The exercises have a comparison of \( \epsilon \)-outlier and among-\(
    \ell \)-smallest to the notions of the order of a list and the rank
    of elements of a list.
\end{example}

\begin{proposition}
    \label{thm:significance:basethm}
    \begin{enumerate}
        \item
            Fix \( n \).
        \item
            Suppose \( \sigma_0 \) is chosen from a stationary
            distribution \( \pi \) of the reversible Markov chain \( X \).
        \item
            Suppose \( X_1, X_2, \dots, X_n \) is a sample of the Markov
            chain starting from \( X_0 = \sigma_0 \).
        \item
            Suppose the state space has a value function \( v :
            \mathcal {X} \to \Reals \).
        \item
            Let
            \[
                \rho(j; \ell, n) = \Prob{v(X_j) \text{ is among-$\ell$
                -smallest among } v(X_0), \dots, v(X_n) }.
            \]
    \end{enumerate}

    Then
    \[
        \rho(0; \ell, n) \le \sqrt{ \frac{2\ell + 1}{n+1}}.
    \]


    % Then the probability that the label of \( X_0 = \sigma_0 \) is an \(
    % \epsilon \)-outlier from among the list of labels observed in the
    % trajectory \( X_0, X_1 , X_2 , \dots X_n \) is, at most, \( \sqrt{2\epsilon}
    % \).  In other words
    % \[
    %     \Prob{ \card{\setof{i}{\alpha_i \le \alpha_0}} \le \epsilon(n+1)}
    %     \le \sqrt{2 \epsilon}.
    %   \]

    %   Let \( X = X_0, X_1, \dots \) be a
    % reversible Markov chain with a stationary distribution \( \pi \),
    % and suppose the states of \( X \) have real-valued labels.  If \( X_0
    % \sim \pi \), then for any fixed \( n \), the probability that the
    % label of \( X_0 \) is an \( \epsilon \)-outlier from among the list
    % of labels observed in the trajectory \( X_0, X_1 , X_2 , \dots X_n \)
    % is, at most, \( \sqrt{2\epsilon} \).
\end{proposition}

\begin{remark}
    The Proposition makes no assumptions on the structure of the Markov
    chain beyond reversibility.  Note that if a Markov chain can be
    broken down into two disjoint communicating classes, i.e.\ two
    disconnected Markov chains and if the individual Markov chains are
    reversible, then so is the full chain, since it is impossible to
    cross from one to the other.  This is a worthwhile observation,
    since all examples of reversible Markov chains so far are
    irreducible, giving the impression that all reversible Markov chains
    are irreducible.

    In particular, the Proposition applies even if the chain is not
    irreducible, i.e.\ even if the state space is not connected
    under the Markov chain.  In this disconnected case the chain will never
    completely mix since it remains in one component. This remark is
    important for the example application below of Markov chains
    sampling valid political districting, where it is not known that the
    Markov chains are irreducible.  Even if valid districting only
    requires contiguous districts of roughly equal populations, it is
    unknown if any valid districting is reachable from any other by a
    legal sequence of steps.  Note that Proposition~%
    \ref{thm:significance:basethm} can only show that a given sample is
    highly unlikely to have been chosen from \( \pi \).  The Proposition
    does not give a way for generating samples from \( \pi \).

    This suggests that a careful choice of label function can help to
    distinguish irreducible components, especially in political districting. Suppose
    that one component is relatively small and unusual compared to more
    typical states from a larger irreducible component.  A label
    function that measures this difference would be a candidate for the
    Proposition.
\end{remark}

\begin{proof}
    \begin{enumerate}
        \item
            Let \( \pi \) denote the stationary distribution for \( X \)
            and suppose that the initial state \( X_0 \) is distributed
            as \( X_0 \sim \pi \), so that in fact, \( X_i \sim \pi \)
            for all \( i \).
        \item
            Consider the set of labels, \( v(X_0), v(X_1), \dots, v(X_n)
            \).  For simplicity, say \( X_j \) is among-\( \ell \)-smallest
            among \( X_0, \dots, X_n \) instead of referring to the
            labels of the states. For a first understanding of the
            proof, focus on the \( \ell = 0 \) case, so the minimum
            value is the focus.
        \item
            For \( 0 \le j \le n \)
            define
            \[
              \rho(j; \ell, n) = \Prob{X_j \text{ is
                  among-$\ell$-smallest from } X_0, \dots, X_n }
            \] and
            \[
                \rho(j; \ell, n \given \sigma) = \Prob{X_j \text{ is
                among-$\ell$-smallest from } X_0, \dots, X_n%
                \given X_j = \sigma }.
            \] Observe that \( \rho(j; 0 ,n) \) is the probability that \(
            \sigma_j \) has the unique minimum value in the Markov chain
            of length \( n \).  Likewise, \( \rho(j; 0, n \given \sigma)
            \) is the probability that \( X_j = \sigma \) has the unique
            minimum value in the Markov chain of length \( n \) given
            that the Markov passes through \( \sigma \) at step \( j \).
        \item
            Observe that because \( X_s \sim \pi \) for all \( s \),
            additionally
            \begin{multline*}
              \rho(j; \ell, n \given \sigma) =\\
              \Prob{X_{s+j} \text{ is
                among-$\ell$-smallest among } X_s, \dots, X_{s+n}
              \given X_j = \sigma }.
            \end{multline*}
        \item\label{enum:significance:basethm5} 
            Because \( X = X_0 , X_1, \dots \) is stationary and
            reversible, the probability that \( (X_0, \dots, X_n ) = (\sigma_0,
            \dots , \sigma_n ) \) is equal to the probability that \( (X_0,\dots,
            X_n ) = (\sigma_n, \dots, \sigma_0) \) for any fixed
            sequence \( (\sigma_0, \dots, \sigma_n ) \).  Thus, any
            sequence \( (\sigma_0,\dots, \sigma_n ) \) for which \(
            \sigma_j = \sigma \) and \( \sigma_j \) is a among-\( \ell \)-smallest
            corresponds to an equiprobable sequence \( (\sigma_n,\dots,
            \sigma_0 ) \), for which \( \sigma_{n-j} = \sigma \) and \(
            \sigma_{n-j} \) is among-\( \ell \)-smallest.  Thus a first
            fact about \( \rho(j; \ell, n \given \sigma) \) is:
            \[
                \rho(j; \ell, n \given \sigma) = \rho(n-j; \ell, n
                \given \sigma).
            \]
        \item \label{enum:significance:basethm6}
            The next claim is:
            \[
                \rho(j; 2\ell, n \given \sigma) \ge \rho(j; \ell, j \given
                \sigma) \cdot \rho(0; \ell, n-j \given \sigma).
            \] The proof of the claim is to consider the sub-event that \(
            X_j \) is among-\( \ell \)-smallest among \( X_0, \dots,
            X_j \) and also among-\( \ell \)-smallest among \( X_j, \dots, X_n \).  These events are
            conditionally independent when conditioning on the value of \(
            X_j = \sigma \), and \( \rho(j; \ell, j \given \sigma) \)
            gives the probability of the first of these events, whereas
            applying the previous step with \( s = j \) shows that \(
            \rho(0, \ell, n-j \given \sigma) \) gives the probability of
            the second event.  Finally, as a sub-event when both of these events
            happen, \( X_j \) is among-\( 2\ell \)-smallest from \( X_0,
            \dots, X_n \).  This establishes the inequality in the claim.
          \item \label{enum:significance:basethm7}
            Combining the information
            \begin{align*}
                \rho(j; 2\ell, n \given \sigma) &\ge \rho(j; \ell, j
                \given \sigma) \cdot \rho(0; \ell, n-j \given \sigma) \\
                &= \rho(0; \ell, j \given \sigma) \cdot \rho(0; \ell,
                n-j \given \sigma) \\
                &= \left( \rho(0; \ell, n \given \sigma) \right)^2.
            \end{align*}
            where the first inequality comes from
            step~\ref{enum:significance:basethm6}. The middle
            equality comes from 
            step~\ref{enum:significance:basethm5}.
            The last inequality comes from the simple observation that
            \( \rho(j; \ell, n \given \sigma) \) is nonincreasing in
            \( n \) for fixed \( j \), \( \ell \) and \( \sigma \), in
            particular for \( j \) fixed to be \( 0 \).
        \item\label{enum:significance:basethm8}
            Now \( \rho(j;\ell,n) = \E{\rho(j; \ell, n \given X_j)} \),
            where the expectation is taken over the random choice of \(
            X_j \sim \pi \).  Thus, taking expectations over the
            inequality in step~\ref{enum:significance:basethm7}
            \begin{multline}
                \rho(j; 2\ell, n) = \E{\rho(j;
                \ell, n \given \sigma)} \ge \E{\rho(j; \ell, n \given
                \sigma)^2} \\
                \ge \left( \E{\rho(0, \ell, n \given \sigma)}\right)^2 =
                \left( \rho(0; \ell, n) \right)^2
            \end{multline}
            where the second of the two inequalities is the
            Cauchy-Schwartz inequality.
        \item
            To complete the proof, sum the left- and right-hand sides
            of the inequality in step~\ref{enum:significance:basethm8} to obtain
            \[
                \sum\limits_{j=0}^n \rho(j; 2\ell, n) \ge (n+1) (\rho(0;
                \ell, n))^2.
            \]
        \item
            Letting \( \xi_j \), \( 0 \le i \le n \) be the indicator
            variable that is \( 1 \) whenever \( X_j \) is among-\( 2\ell
            \)-smallest among \( X_0, \dots , X_n \), then \( \sum_{\nu=0}^n
            \xi_{\nu} \) is the number of among-\( 2\ell \)-smallest
            terms which is at most \( 2\ell + 1 \).  Therefore,
            linearity of expectation gives that
            \[
                2\ell + 1 \ge (n + 1)(\rho(0; l,n) )^2
            \] giving
            \[
                \rho(0; \ell, n) \le \sqrt{ \frac{2\ell + 1}{n+1}}.
            \]
    \end{enumerate}
\end{proof}

\begin{corollary}[The $\sqrt{\epsilon}$ test]
    \label{thm:significance:sqrtepstest}
    \begin{enumerate}
        \item
            Fix \( n \).
        \item
            As a null hypothesis, suppose \( \sigma_0 \) is chosen from
            a stationary distribution \( \pi \) of the reversible Markov
            chain \( X \).
        \item
            Suppose \( X_1, X_2, \dots, X_n \) is a sample of the Markov
            chain starting from \( X_0 = \sigma_0 \).
        \item
            Suppose the state space has a value function \( v :
            \mathcal {X} \to \Reals \).
    \end{enumerate}
    Then the event that \( v(X_0) \) is an \( \epsilon \)-outlier among \(
    v(x_0), \dots v(x_n ) \) is significant at \( p = \sqrt{2\epsilon} \)
    under the null hypothesis that \( X_0 \sim \pi \).
\end{corollary}
\index{$\sqrt{\epsilon}$-test}

\begin{proof}
    \begin{enumerate}
        \item
            This is a restatement of Proposition~%
            \ref{thm:significance:basethm} as a statistical test in
            terms of the proportion \( \epsilon \).
        \item
            Let the reversible Markov chain \( X \), the stationary
            distribution \( \pi \), the labeling \( v \), fixed integer \(
            n \), and fixed small proportion \( \epsilon \) determine a \(
            n \)-vector \( (\rho(0; \epsilon, n), \rho(1; \epsilon, n),
            \dots, \rho(n; \epsilon, n)) \) where, with a slight abuse
            of notation, for each \( i \), \( \rho(i; \epsilon, n) \) is
            the probability that in a \( n \)-step stationary trajectory
            \( X_0, X_1, \dots X_n \), \( v(X_i) \) is
            among-\(\lfloor \epsilon(n+1) \rfloor)-smallest
            in the list \( v(X_0), v(X_1), \dots,
            v(X_n) \).
        \item
            Proposition~\ref{thm:significance:basethm} proves
            \[
                \rho(0; \epsilon, n) \le \sqrt{\frac{2 \lfloor \epsilon
                (n+1) \rfloor}{n+1}} \le \sqrt{\frac{2 \epsilon (n+1)}{n+1}}
                = \sqrt{2\epsilon}.
            \]
    \end{enumerate}
\end{proof}

\begin{remark}
    Chikina, et al.
    \cite{Chikina2860} call Corollary~%
    \ref{thm:significance:sqrtepstest} the \( \sqrt{\epsilon} \) test,
    and this section follows their terminology.  However, more logically
    it could be called the \( \epsilon \) test, referring to the outlier
    fraction, or the \( \sqrt{2 \epsilon} \) test referring to the
    significance level.
\end{remark}

\begin{remark}
    The significance level is often called the Type I error rate.%
    \index{Type I error rate}
    It is the probability of rejecting the null hypothesis, given that
    the null hypothesis is true.  Often in statistics, the probability
    of rejecting the null hypothesis is denoted by \( \alpha \).  So
    here Corollary~%
    \ref{thm:significance:sqrtepstest} says that if the label is an \(
    \epsilon \)-outlier, then the probability of \emph{erroneously}
    concluding that \( X_0 \) did \emph{not} come from the stationary
    distribution, when in fact \( X_0 \) \emph{did} come from the stationary
    distribution, is less than \( \sqrt{2 \epsilon } \).
\end{remark}

\begin{remark}
    The power%
    \index{power}
    of a binary hypothesis test is the probability that the test
    correctly rejects the null hypothesis when the alternative
    hypothesis is true.  Here that would be the probability that the \(
    \sqrt{\epsilon} \) test correctly says \( X_0 \) as an \( \epsilon \)-outlier
    did \emph{not} come from the stationary distribution Later sections
    consider the statistical power of the \( \sqrt{\epsilon} \) test and
    show that the relationship \( p \approx \sqrt{\epsilon} \) is best
    possible.  It is currently an open question whether the constant \(
    2 \) can be improved.
\end{remark}

Another way to look at the theorem is the following.  The Markov chain \(
\mathcal{X} \), the stationary distribution \( \pi \), the labeling \( v
\), and a fixed integer \( n \) determine a \( n \)-vector \( (p_0^n, p_1^n,
\dots, p_n^n) \) where for each \( \nu \), \( p_{\nu}^n \) is the
probability that in a \( n \)-step stationary trajectory \( X_0, X_1,
\dots X_n \), the minimum \( v \) label value occurs at \( X_i \).  In
other words, choosing \( X_0 \) randomly from the stationary
distribution \( \pi \) and taking \( n \) steps in \( \mathcal{X} \) to
obtain the trajectory \( X_0, X_1, \dots X_n \), the probability that
observing \( v(X_i) \) is the minimum among \( v(X_0), v(X_1), \dots, v(X_n)
\) is \( p_i^n \).  Adopt the convention that ties among the values \( v
(X_0), v(X_1), \dots, v(X_n) \) picks \( p_i^n \) randomly so that \( p_0^n
+ p_1^n + \cdots + p_n^n = 1 \) for \( X \), \( \pi \), \( n \).

Thinking that the minimum value is roughly equally likely to occur
anywhere in the chain, it might be natural to guess that \( p_i^n
\approx \frac{1}{n+1} \). To be the minimum value means that the value
occurs in the least \( \frac{1}{n+1} \) fraction, so \( \epsilon = \frac
{1}{n+1} \). Corollary~%
\ref{thm:significance:sqrtepstest} shows that \( p_0^n \le \frac{2}{\sqrt
{%
n+1}} \).  The next section has an example that \( p_0^n \) can be as
large as \( \frac{1}{\sqrt{2\pi n}} \).  The example means that the \( O
(\sqrt{\epsilon}) \) probability from Corollary~%
\ref{thm:significance:sqrtepstest}, as the worst possible behavior for \(
p_0^n \), can actually be achieved.

In the general setting of a reversible Markov chain, the Corollary leads
to a simple quantitative procedure for asserting rigorously that \(
\sigma_0 \) is atypical with respect to \( \pi \) without knowing the
mixing time of \( X \):  simply observe a random trajectory \( \sigma_0
= X_0 , X_1, X_2, \dots, X_n \) from \( \sigma_0 \) for any fixed \( n \).
If \( v(\sigma_0 ) \) is an \( \epsilon \)-outlier among \( v(X_0), v(X_1),
\dots, v(X_n) \), then the probability of \emph{erroneously} concluding
that \( X_0 \) did \emph{not} come from the stationary distribution,
when in fact the null hypothesis that \( X_0 \) did come from the
stationary distribution is true, is less than \( \sqrt{2 \epsilon } \).

Corollary~%
\ref{thm:significance:sqrtepstest} leads to the Outlier Test
Algorithm:%
\index{Outlier Test Algorithm}
\begin{algorithm}[H]] \KwData{Markov chain $X$, starting state
    $\sigma_0$, value function $v(\cdot}$}
\KwResult{probability
  $\sqrt{2 \epsilon}$ of \emph{erroneously} concluding that \( X_0 \)
  did \emph{not} come from the stationary distribution}
        Begin from the state being evaluated for ``outlier'' status.\;
        Make a sequence of $n$ random changes to the states according to the
        Markov chain transition probabilities.\;
        Evaluate the labeling of the chain.\;
        Call the original state an ``outlier'' if the overwhelming
        majority of labels (more than \( 1 - \epsilon \) of the labels)
        are greater in label than the starting state.\;
    \caption{Outlier Test Algorithm.}    
\end{algorithm}

The Outlier Test Algorithm is useful because \( \sqrt{2\epsilon} \) is
reasonably close to \( 0 \) for \( \epsilon \) small; in particular, it
is possible to obtain ``good enough'' statistical significance from
observations which can be made with reasonable computational resources.
Of course, smaller significance compared to \( \epsilon \) would be even
better, but, as already noted, \( p = O(\sqrt{\epsilon}) \) is the best
possible.

Apart from its application to gerrymandering below, the \( \sqrt{\epsilon}
\) Test has a simple informal interpretation for the general behavior of
reversible Markov chains, namely:  long-term typical (i.e., stationary)
states are unlikely to change in a measurable way (as measured by the
label) under a sequence of chain transitions, with a best-possible
quantification of this fact (up to constant factors).

Next is a corollary of Theorem~%
\ref{thm:significance:basethm} that applies to the setting where \( X_0
\) is not distributed as a stationary distribution \( \pi \) but instead
is distributed with small total variation distance to \( \pi \).

\begin{corollary}
    \begin{enumerate}
        \item
            Let \( X \) be a reversible Markov chain on \( \mathcal{X} \).
        \item
            Let \( v :  \mathcal{X} \to \Reals \) be a label function.
        \item
            Let \( X_0 \) be a state chosen from \( \mathcal{X} \) with
            distribution such that \( \| X_0 - \pi \|_{TV} \le \epsilon_1
            \).
    \end{enumerate}
    Then for any fixed \( n \), the probability that the label of \( X_0
    \) is an \( \epsilon \)-outlier from among the list of labels
    observed in the trajectory \( X_0, X_1, \dots X_n \) is at most \(
    \sqrt{2\epsilon} + \epsilon_1 \)
\end{corollary}

\begin{proof}
    \begin{enumerate}
        \item
            If \( \rho_1 \), \( \rho_{2} \), and \( \tau \) are
            probability distributions, then the product distributions \(
            (\rho_1 , \tau ) \) and \( (\rho_2 , \tau ) \) satisfy \( \|
            (\rho_1 , \tau )- (\rho_2, \tau )\|_{TV} = \|\rho_1- \rho_2
            \|_{TV} \).
        \item
            The plan is to split the randomness in the trajectory \( X_0,
            \dots , X_n \) of the Markov chain into two independent
            sources:  the initial distribution is \( X_0 \sim \rho \),
            and \( \tau \) is the uniform distribution on sequences of
            length \( n \) of real numbers \( r_1 , r_2, \dots, r_n \)
            in \( [0, 1] \).
        \item
            The distribution of the trajectory \( X_0 , X_1,\dots, X_n \)
            is the product \( (\rho, \tau ) \) by using sequences of
            reals \( r_1, \dots, r_n \) to choose transitions in the
            chain; from \( X_i = \sigma_i \), if there are \( L \)
            transitions possible, with probabilities \( p_1, \dots, p_L \).
        \item
            Then, make the \( t \)th possible transition if \( r_i \in [p_1
            + \cdots + p_{t-1}, p_1 + \cdots + p_{t-1 }+ p_t] \).  If \(
            \|\rho- \pi\|_{TV} \le \epsilon_1 \), then \( \|(\rho, \tau
            )- (\pi, \tau )\|_{TV} \le \epsilon_1 \).
        \item
            Therefore, any event that would happen with probability at
            most \( p \) for the sequence \( X_0 ,\dots, X_n \) when \(
            X_0 \sim \pi \) must happen with probability at most \( p +
            \epsilon_1 \) when \( X_0 \sim \rho \), where \( \|\rho- \pi\|_
            {TV} \le \epsilon_1 \).  The corollary follows.
    \end{enumerate}
\end{proof}

\begin{example}
    Consider a specific simple case of the alternative Ehrenfest urn
    model.%
    \index{Ehrenfest urn model}%
    Two urns, labeled \( A \) and \( B \), contain a total of \( N = 7 \)
    balls.  At each step a ball is selected at random with all selections equally
    likely.  Then an urn is selected, urn \( A \) with probability \( p
    = \frac{1}{2} \) and urn \( B \) with probability \( q = 1-p = \frac
    {1}{2} \) and the ball is moved to that urn.  The state at each step
    is the number of balls in the urn \( A \), from \( 0 \) to \( N \).
    Start from state \( 0 \), with no balls in urn \( A \).  This Markov
    chain is ergodic.  All states are accessible and all states
    communicate.  All states are recurrent and there are no transient
    states.  The chain is periodic with period \( 1 \) so it is
    regular.  The standard Ehrenfest urn model, without the random
    choice of urns at each step, is periodic with period \( 2 \). 
    Having period \( 1 \) makes interpretation of the results simpler.

    For this very simple example, let the label function be the number
    of balls in urn \( A \), so that the label function is identical
    with the state number.  The simplicity of the alternative Ehrenfest
    urn model limits the possibilities for the labels.  More interesting
    label functions appear in the gerrymandering example later, where
    the states have considerably more structure and information
    associated with them.

    The second largest (in absolute magnitude) eigenvalue of the
    transition probability matrix is approximately \( 0.857 \).  The
    total variation distance of the probability distribution of the
    states at step \( n \) should differ from the stationary
    distribution by a multiple of \( (0.857)^n \).  Therefore, it takes
    about \( 31 \) steps for the total variation distance of the
    probability distribution of the states to differ from the stationary
    distribution by less than \( 0.01 \).  For the alternative Ehrenfest
    urn model with \( 7 \) balls, the stationary distribution of the
    Markov chain started from state \( 0 \) will be in state \( 0 \)
    about \( 0.78125\% \) of the steps.  In \( 199 \) steps, together
    with the starting state \( 0 \), it would be normal to see the state
    \( 0 \) just once or twice.  Equivalently, the mean recurrence time
    to state \( 0 \) is \( 128 \) steps, so again in \( 199 \) steps it
    would be normal to see the state \( 0 \) just once or twice.  This
    shows that the state \( 0 \) is an atypical state with respect to
    the stationary distribution, which should be expected from the
    statistical mechanical interpretation of the alternative Ehrenfest
    urn model. The state \( 0 \) is a state of high
    statistical-mechanical order, with all the balls in urn \( B \),
    none in urn \( A \).  Such highly ordered states should be unlikely.
    In summary, the label \( 0 \) should be fairly unusual, in spite of
    the fact that \( 0 \) is a state in the stationary distribution.

    If \( \epsilon = 0.01 \), then this analysis predicts that value \(
    0 \) would usually be an \( \epsilon \)-outlier for the set of
    values generated by the chain, since it would usually appear in the
    smallest \( \epsilon \)-fraction of the ranked values, that is, the
    least two values.  Equivalently \( 198 \) of the \( 200 \) values
    from the path would be greater than \( 0 \).  Actually running
    simulations of the alternative Ehrenfest Markov chain shows that state \( 0 \)
    is an \( 0.01 \)-outlier according to this test about half of the
    time, see the exercises.  The simulations show that rejecting the
    null hypothesis, that is,  concluding in error that \( 0 \) is \emph{not}
    unusual with respect to the stationary distribution, occurs about
    half the time.  

    However, with just \( 8 \) labels, there is not much to distinguish
    being an \( \epsilon \)-outlier, and variation among Markov chain
    runs could account for false negatives, that is, finding that the
    truly unusual state \( 0 \) (unusual in the sense of high
    statistical-mechanical order and unusual in the sense of
    infrequent visitation) occurs in the larger \( 1 - \epsilon \)
    fraction of the labels, asserting that label \( 0 \) is not unusual.
    Experiments with more trials with longer path lengths for the Markov
    chain shows that the proportion of trials showing that \( 0 \) is an
    \( \epsilon \)-outlier, that is, the empirical probability, slowly
    increases.  Accepting the null hypothesis increases with
    increasing the length of the chain, as expected.  This
    illustrates that Corollary~%
    \ref{thm:significance:sqrtepstest} bounds the Type I error, not the
    Type II error.  This is an illustration of the problem of determining
    the statistical power of the test discussed later.

    Considering larger specific cases of the alternative
    Ehrenfest urn model%
    \index{Ehrenfest urn
  model}%
.   with \( N = 21 \) and \( N = 51 \) balls illustrates the utility of a more discriminating
    label function.  With \( N = 21 \) balls, Markov chain runs of \( 399 \) steps in \(
    400 \) trials the proportion showing that \( 0 \) is an \( \epsilon
    \)-outlier is about \( 0.92 \).  With longer Markov chain runs of \(
    800 \) or \( 1600 \) steps in \( 1000 \) trials the proportion showing that \( 0 \)
    is an \( \epsilon \)-outlier is better than \( 0.97 \).  With \( N
    = 51 \) balls, the probability of showing state \( 0 \) is unusual
    in long runs is virtually certain.

    The important observation is that the \( \epsilon \)-outlier test
    can show that state \( 0 \) is unusual with high confidence
    without knowing the mixing time, or that the Markov chain is in
    the steady state, or even the steady state distribution.

\end{example}

\subsection*{Example Where \( p = O(\sqrt{\epsilon}) \) is the Best
Possible}

It might be natural to suspect that observing \( \epsilon \)-outlier
status for \( \sigma \) on a random trajectory from \( \sigma \) is
significant at something like \( p = O(\epsilon) \) instead of the
significance \( p = O(\sqrt{\epsilon}) \) established by Corollary~%
\ref{thm:significance:sqrtepstest}.  However, because Corollary~%
\ref{thm:significance:sqrtepstest} places no demand on the mixing rate
of \( X \), it should instead seem remarkable that any significance can
be shown in general.  An example in this section shows that
significance at \( p = O( \sqrt{\epsilon}) \) is the best possible.

Let \( N \) be a large integer.  Let \( X \) be the Markov chain where \(
X_0 \) is chosen at random uniformly in \( \mathcal{X} = \set{0, 1, 2, \dots , N-1} \),
and for any \( \nu \ge 1 \), \( X_{\nu} \) is equal to \( X_{\nu-1} +
\zeta_{\nu} \) modulo \( N \), where \( \zeta_{\nu} = \pm 1 \) with
equal probability \( \frac{1}{2} \).  That is, \( X \) is random walk on
the circle.  Note that the chain is stationary and reversible.  Let the
value function be the position around the circle, that is, \(
v(X_{\nu}) = X_{\nu} \).

Given \( n \) and \( \epsilon \), choose \( N > \frac{2n+1}{\epsilon} \).
Then \( N \) is large enough that the distance of \( X_0 \) from \( 0 \)
in both directions is greater than \( n \) with probability greater
than \( 1 -
\epsilon \). Conditioning on this event, we have that \( X_0 \) is
minimum among \( X_0, \dots , X_n \) if and only if all of the partial
sums \( \sum\limits_{\nu=1}^ {j} \zeta_\nu \) are positive for \( j \le
n \).  The probability of this event is just the probability that an \( n
\)-step \( 1 \)-dimensional random walk from \( X_0 \) takes a first
step to the right and does not return to the origin.  The calculation of
this probability is a classical problem in random walks, 
solved using the reflection principle.  In particular, for \( n \) even,
the probability is given by (see the exercises) 
\[
    \frac{1}{2^{n+1}}\binom{n}{n/2} \asympt \frac{1}{\sqrt{2 \pi n}}.
\] Because being the minimum of \( X_0, \dots , X_n \) corresponds to
being an \( \epsilon \)-outlier for \( \epsilon = \frac{1}{n+1} \) this
example shows that the probability of being an \( \epsilon \)-outlier
can be as high as \( \sqrt{\epsilon/(2\pi)} \).  Of course, \( \sqrt{\epsilon/
(2\pi)} < \sqrt{2\epsilon} \) so the best possible value of the constant
in the \( \sqrt{\epsilon} \) test is an interesting question.

Figure~\ref{fig:serialsignificance:circlewalk} schematically illustrates a specific
example.  Let $n = 198$, $\epsilon = 0.01$, so $N >
\frac{2n+1}{\epsilon} = 39,700$.  Take $N = 40{,}000}$ for
simplicity.  States, such as $20{,}000$, chosen outside the region
marked with bars have the distance of \( X_0 \) from \( 0 \)
in both directions is greater than \( n = 198 \) with probability greater
than \( 1 - \epsilon = 0.99 \).  An \( n
\)-step random walk from \( X_0 \) satisfying the condition is
drawn schematically.

\begin{figure}
  \centering
  \begin{asy}
settings.outformat = "pdf";

import geometry;

size(5inches);

real myfontsize = 12;
real mylineskip = 1.2*myfontsize;
pen mypen = fontsize(myfontsize, mylineskip);
defaultpen(mypen);

point O = (0,0);
real r = 2;
circle C = circle(O,r);
draw(C);

dot("$0$", relpoint(C, 0.25), N);
dot("$1$", relpoint(C, 0.24), N);
dot("$2$", relpoint(C, 0.23), N);

dot(relpoint(C, 0.21));
dot(relpoint(C, 0.20));
dot(relpoint(C, 0.19));

dot("$199$", relpoint(C, 0.15), NE);

dot("$39{,}999$", relpoint(C, 0.26), S);
dot("$39{,}998$", relpoint(C, 0.27), NW);

dot(relpoint(C, 0.29));
dot(relpoint(C, 0.30));
dot(relpoint(C, 0.31));

dot("$39{,}801$", relpoint(C, 0.35), NW);

arc avoided = arc(C, 51, 129);
draw(avoided, Bars);

dot("$20{,}000$", relpoint(C, -0.25), S);
dot("$19{,}999$", relpoint(C, -0.24), NE);
dot("$20{,}001$", relpoint(C, -0.26), NW);

dot(relpoint(C, 0.00));
dot(relpoint(C, -0.01));
dot(relpoint(C, 0.01));

dot(relpoint(C, 0.50));
dot(relpoint(C, 0.49));
dot(relpoint(C, 0.51));

circle C1 = circle(O, 0.92*r);
circle C2 = circle(O, 0.91*r);
circle C3 = circle(O, 0.90*r);

arc walk1 = arc(C1, -90, -130, direction=CW);
arc walk2 = arc(C2, -130, -110, direction=CCW);
arc walk3 = arc(C3, -110, -140, direction=CW);

draw(walk1, Arrow);
draw(walk2, Arrow);
draw(walk3, Arrow);
\end{asy}
  \caption{A schematic diagram of the random walk on a circle
    illustrating the example of the best possible bound.}
  \label{fig:serialsignificance:circlewalk}
\end{figure}
\subsection*{Notes on Statistical Power}
% This section based on pnas.201617540SI.pdf, Section S7.

The effectiveness of the \( \sqrt{\epsilon} \) test depends on the
availability of a sufficiently discriminating choice for the value
function \( v \) and the ability to run the test for long enough (in
other words, choose \( n \) large enough) to detect that the presented
state is a local outlier.  It is possible, however, to make a general
statement about the power of the test when \( n \) is chosen large
relative to the actual mixing time of the chain.  Recall that one
potential application of the test is in situations where the mixing time
of the chain is actually accessible through reasonable computational
resources, although this fact cannot be proved rigorously, because
theoretical bounds on the mixing time are not available.  In particular,
we do know that the test is very likely to succeed when \( n \) is
sufficiently large and \( v(\sigma_0) \) is atypical.

% The following may be a relevant reference for the previous remark.
% MIXING TIME ESTIMATION IN REVERSIBLE MARKOV CHAINS
% FROM A SINGLE SAMPLE PATH
% DANIEL HSU, A RYEH KONTOROVICH, D AVID A. LEVIN,
% YUVAL PERES, C SABA SZEPESVARI AND GEOFFREY WOLFER

% The Annals of Applied Probability
% 2019, Vol. 29, No. 4, 2439 - 2480
% https://doi.org/10.1214/18-AAP1457
% Institute of Mathematical Statistics, 2019

\begin{remark}
    Let \( \tau_{\text{rel}} \) be the relaxation time%
    \index{relaxation time}
    for \( X \) defined as \( 1/ (1-\lambda_*) \) where
    \[ \lambda_* =
    \max{\abs{\setof{\lambda}{\text{$\lambda$ is an eigenvalue of $P$,
            $\lambda \ne 1$}}}}
    \]
    is the \emph{absolute spectral gap} for \( X \). The relaxation time is
    related to the mixing time of \( X \).  Let \( d(t) =
    \sup_{\sigma} \| P^t - \pi \|_{TV} \).  The \emph{mixing time} is
    defined as \( \tau_{\text{mix}}(\epsilon) = \min\setof{t}{d(t) \le
      \epsilon} \).  Then, \cite[Theorems 12, 3, 12.4]{levin09},
    \[
      (\tau_{\text{rel}}-1) \log \left( \frac{1}{2\epsilon} \right)
      \le \tau_{\text{mix}} \le \tau_{\text{rel}} \log \left(
        \frac{1}{\epsilon \min_x \pi(x)} \right).
    \]
    In some cases the relaxation time can be calculated from the
    eigenvalues of the transition probability matrix to give bounds on
    the mixing time, which is difficult to determine directly.
\end{remark}

Theorem~%
\ref{thm:significance:powerthm} below is a simple application of the
following theorem of Gillman
\cite[Theorem 2.1]{gillman98} translated into the current notation and
assumptions.

\begin{theorem}[Gillman's Chernoff-Type Bound]
    \label{thm:significance:gillman}
    \begin{enumerate}
        \item
            Let \( X = X_0, X_1, \dots \) be a reversible Markov chain
            on \( \mathcal{X} \).
        \item
            Let \( A \subset \mathcal{X} \).
        \item
            Let \( N_n(A) \) denote the number of visits to \( A \)
            among \( X = X_0, X_1, \dots, X_n \).
    \end{enumerate}
    Then for any \( \gamma > 0 \),
    \[
        \Prob{N_n(A)/n - \pi(A) \ge \gamma} \le \left( 1 + \frac{\gamma}
        {10 \tau_2} \right) \sqrt{\sum_{\sigma} \frac{(\Prob{X_0 =
        \sigma})^2}{\pi (\sigma)}} \exp \left( \frac{-\gamma^2 \cdot n}{20
        \tau_2} \right).
    \]
\end{theorem}
\index{Gillman's Chernoff-Type Bound}

\begin{remark}
  Note that the application of Gillman Theorem 2.1 quoted in
  pnas.201617540SI.pdf, Section S7 appears to have some typographical
  errors, specifically,
  \begin{enumerate}
  \item The use of \( n \) in the event on the left side and the \( n
    \) appearing in the numerator of the exponential function argument
    should both be \( n \), following to the notation in
    pnas.201617540SI.pdf, Section S7.
  \item The \( n \) appearing in the numerator of the
    large-parenthesized coefficient should not appear.
  \item These typographical errors have been corrected in the
    statement of above.
  \item These typographical errors do ot affect the overall
    conclusions related to the theorem.
  \end{enumerate}
\end{remark}

\begin{theorem}
    \label{thm:significance:powerthm}
    \begin{enumerate}
        \item
            Let \( X \) be a reversible Markov chain on \( \mathcal{X} \).
        \item
            Let \( v :  \mathcal{X} \to \Reals \) be a label function.
        \item
            Given \( \sigma_0 \), suppose that for a random state \(
            \sigma \sim \pi \),
            \[
                \Prob{v(\sigma) \le v(\sigma_0)} \le \epsilon.
            \]
    \end{enumerate}
    Then with probability at least
    \[
        \rho \ge 1 - \left( 1 + \frac{\epsilon}{10 \tau_2} \right)
        \cdot \frac{1}{\sqrt{\pi_{\min}}} \cdot \exp \left( \frac {-\epsilon^
        {2} \cdot n}{20 \tau_{2}} \right)
    \] \( v(\sigma) \) is a \( 2\epsilon \)-outlier among \( v(\sigma_0),
    v(\sigma_1), \dots, v(\sigma_n) \) where \( \sigma_0, \sigma_1,
    \dots, \sigma_n \) is a random trajectory starting from \( \sigma_0 \).
\end{theorem}

\begin{proof}
  \begin{enumerate}
  \item 
    Apply Theorem~%
    \ref{thm:significance:gillman} with \( A \) as the set of states \(
    \sigma \in \mathcal{X} \) such that \( v(\sigma) \le v(\sigma_0) \),
    \( X_0 = \sigma_0 \) and \( \gamma = \epsilon \).  By assumption \(
    \pi(A) \le \epsilon \).
  \item  Theorem~%
    \ref{thm:significance:gillman} gives that
    \[
        \Prob{N_{n}(A)/n > 2\epsilon} \le \left( 1 + \frac{\epsilon n}{10
        \tau_2} \right) \cdot \frac{1}{\sqrt{\pi_{\min}}} \cdot \exp
        \left( \frac{-\epsilon^{2 n}}{20 \tau_2} \right).
      \]
    \item 
      The statement of the proof uses the additional simplification:
      \[
        \frac{(\Prob{X_0 =
        \sigma})^2}{\pi (\sigma)}} \le \frac{(\Prob{X_0 =
      \sigma})^2}{\pi_{\min}}}
     \]
     so
     \[
       \sum_{\sigma} \frac{(\Prob{X_0 =\sigma})^2}{\pi (\sigma)}}
     \le
     \frac{\sum_{\sigma}{(\Prob{X_0 =\sigma})^2}}{\pi_{\min}}
     \le
     \frac{(\sum_{\sigma}{\Prob{X_0 =\sigma})^2}}{\pi_{\min}}
     = \frac{1}{\pi_{\min}}.
   \]
   using standard inequalities.
  \end{enumerate}
\end{proof}

\begin{remark}
    The probability \( \rho \) in Theorem~%
    \ref{thm:significance:powerthm} converges in \( n \) exponentially
    quickly to \( 1 \).  The probability \( \rho \)is very close to \( 1
    \) if \( n \) is large relative to \( \tau_2 \).  In particular,
    Theorem~%
    \ref{thm:significance:powerthm} shows that the \( \sqrt{\epsilon} \)
    test will work when the test runs long enough.  Of course, one
    strength of the \( \sqrt{\epsilon} \) test is that it can sometimes
    show bias, even when \( n \) is far too small to mix the chain,
    which is almost certainly the case for the application to
    gerrymandering later.  When these short-\( n \) runs are successful
    at detecting bias is, of course, dependent on the relationship of
    the presented state \( \sigma_0 \) and its local neighborhood in the
    chain.
\end{remark}


\subsection*{Application to Political Districting}

Pennsylvania is divided into roughly \( 9,000 \) census blocks.  Define
a division of these blocks into \( 18 \) districts to be a valid
districting of Pennsylvania if districts differ in population by less
than \( 2\% \), are contiguous, are simply connected (districts do not
contain holes), and are ``compact''.  Roughly, this final condition
prohibits districts with extremely contorted structure.  The state space
of the Markov chain is the set of valid districtings of Pennsylvania.
One step of the Markov chain consists of randomly swapping a precinct on
the boundary of a district to a neighboring district if the result is
still a valid districting.  The chain is adjusted slightly to ensure
that the uniform distribution on valid districting is indeed a
stationary distribution for the chain.  Observe that this Markov chain
has a potentially huge state space; if the only constraint on valid
districting was that the districts have roughly equal population, there
would be \( 10^{10000} \) or so valid districtings.  (See the exercises
for a justification of this estimate.) Although contiguity and
especially, compactness are severe restrictions that will decrease this
number substantially, it seems difficult to compute effective upper
bounds on the number of resulting valid districtings, and certainly, the
number of valid districtings is still enormous.  Impressively, these
considerations are all immaterial to the very general method of
significance testing. Regarding mixing time, even in chains with
relatively weak constraints on the districtings (and very fast running
time in implementation) seems to mix too slowly to sample \( \pi \),
even heuristically.

Applying the \( \sqrt{\epsilon} \) test involves the choice of a label
function \( v(\sigma) \), which assigns a real number to each
districting.  Two useful label functions are
\begin{itemize}
    \item
        \( v_{\text{var}} \) is the (negative) variance of the
        proportion of Democrats in each district of the districting (as
        measured by 2012 presidential votes), and
    \item
        \( v_{\text{MM}} \) is the difference between the median and
        mean of the proportions of Democrats in each district.
\end{itemize}
The use of the variance is motivated by the fact that it can change
quickly with small changes in districtings.  The median-mean difference \(
v_{\text{MM}} \) is motivated by the fact that this metric has a long
history of use in gerrymandering and is directly tied to the goals of
gerrymandering, An important point is that these label functions are not
based on an assumption that small values of \( v_{\text {var}} \) or \(
v_{\text{MM}} \) directly imply gerrymandering.  Instead, because the \(
\sqrt{\epsilon} \) test is valid for any fixed label function, these
labels are tools used to show significance, which are chosen because
they are simple and natural functions on vectors that can be quickly
computed, seem likely to be different for typical versus gerrymandered
districtings, and have the potential to change relatively quickly with
small changes in districtings.  For chains of length \( 2^{40} \) steps
with various notions of valid districtings differing in compactness
measure and threshold, the \( \sqrt{\epsilon} \) test showed
significance at \( p \) values in the range from \( 10^{-4} \) to \( 10^
{-5} \) for the \( v_{\text {MM}} \) label function and the range from \(
10^{-4} \) to \( 10^{-7} \) for the \( v_{\text{var}} \) label function,
see
\cite{chikina2860si}.

It should also be kept in mind that, although the \( \sqrt{\epsilon} \)
Test gives a method to show that the districting is unusual among
districtings distributed according to the stationary distribution \( \pi
\), it does so without giving a method to sample from the stationary
distribution.  In particular, the method cannot answer the question of
how many seats Republicans and Democrats should have in a \emph{typical}
districting of Pennsylvania, because the chain is not selecting from the
stationary distribution.  Instead, the \( \sqrt{\epsilon} \) gives a way
to disprove the hypothesis \( X_{0} \sim \pi \) without sampling \( \pi \).
Consider the following specialization of the Local Outlier Algorithm to
evaluate political districting of a state:\\
\textbf{Local Outlier Test}%
\index{Local Outlier Test}
\begin{enumerate}
    \item
        Begin with the districting being evaluated.
    \item
        \label{enum:significance:mcstep} Make a sequence of random
        changes to the districting, while preserving some set of
        constraints imposed on the districtings.
    \item
        Evaluate the partisan properties of each districting encountered
        (e.g., by simulating elections using past voting data).
    \item
        \label{enum:significance:crafted} Call the original districting
        ``carefully crafted'' or ``gerrymandered'' if the overwhelming
        majority of districtings produced by making small random changes
        are less partisan than the original districting.
\end{enumerate}

Naturally, the test described above can be implemented so that it
precisely satisfies the hypotheses of the Theorems in this section.  For
this purpose, a (very large) set of comparison districtings are defined,
to which the districting being evaluated belongs.  For example, the
comparison districtings may be the districtings built out of Census
blocks (or some other unit) which are contiguous, equal in population up
to some specified deviation, or include other constraints.  A Markov
chain \( \mathcal{X} \) is defined on this set of districtings, where
transitions in the chain correspond to changes in districtings.  For
example, a transition may correspond to randomly changing the district
assignment of a randomly chosen Census block which currently borders
more than one district, subject to the constraints imposed on the
comparison set.  This is called a flip chain'' on the state space of all
feasible districting.  Alternatively, a ``ReCom chain'' method works as
follows:  merge two districts in the plan, generate a minimum spanning
tree for the precincts in the merged district, then ``split'' the merged
district into two new districts by finding a population-balanced cut of
the minimum spanning tree.  The ``random changes' from Step~%
\ref{enum:significance:mcstep} will then be precisely governed by the
transition probabilities of the Markov chain \( X \).  By designing \( X
\) so that the uniform distribution \( \pi \) on the set of comparison
districtings \( \Sigma \) is a stationary distribution for \( X \),
Theorem 1.1 gives an upper bound on the false-positive rate (in other
words, a global statistical significance) for the ``gerrymandered''
declaration when it is made in Step~%
\ref{enum:significance:crafted}.


\visual{Section Starter Question}{../../../../CommonInformation/Lessons/question_mark.png}
\section*{Section Ending Answer}

Making a Type II error would be to perform the statistical test,
and \emph{accept} the hypothesis that the state is ``unusual'' with
respect to some function labeling the states when in fact, the state
is \emph{not} unusual with respect to the labeling.  That is to
conclude that the state is unusual when in fact it is typical.

Making a Type I error would be to perform the statistical test,
and \emph{reject} the hypothesis that the state is ``unusual'' with
respect to some function labeling the states when in fact, the state
\emph{is} unusual with respect to the labeling.


\subsection*{Sources} This section is adapted from
\cite{Chikina2860, Chikina2019, doi:10.1080/2330443X.2020.1806763}.

% @misc{chikina2019separating,
%       title={Separating effect from significance in Markov chain tests}, 
%       author={Maria Chikina and Alan Frieze and Jonathan Mattingly and Wesley Pegden},
%       year={2019},
%       eprint={1904.04052},
%       archivePrefix={arXiv},
%       primaryClass={math.PR}
%     }
    
The first paragraph of the remark following Proposition~%
\ref{{thm:significance:basethm}} about Markov chains with distinct
irreducible components is adapted from
https://math.stackexchange.com/questions/1636992/an-example-of-a-reversible-but-reducible-markov-chain

The remark about relaxation times and mixing times is adapted from
\cite{levin09} and
https://math.stackexchange.com/questions/1382222/relaxation-time-and-mixing-time-of-markov-chains

\hr

\visual{Algorithms, Scripts, Simulations}{../../../../CommonInformation/Lessons/computer.png}
\section*{Algorithms, Scripts, Simulations}

\subsection*{Algorithm}

\begin{algorithm}[H]
  \SetKwInOut{Input}{input}\SetKwInOut{output}
  \SetKwData{pL}{pathLength}
  \SetKwData{nT}{nTrials}
  \SetKwData{res}{results}
  \SetKwData{oT}{outlierThreshold}
  \SetKwData{neO}{nEpsilonOutlier}
  \SetKwData{pEO}{probEpsOutlier}
  \SetKwFunction{Floor}{floor}
  
  \Input{Number of balls $N$ and probability of urn switch, $p$}
  \Output{Table of empirical probabilities of $\epsilon$-Outlier Test
    success.}
  \BlankLine
  $N \leftarrow 7$, $p \leftarrow 1/2$, $q \leftarrow 1-p$.\;
  \tcp{Build transition $(N+1) \times (N+1)$probability matrix}
  $P_{11} \leftarrow q$, $P_{12} \leftarrow p$\;
  $P_{N+1,N} \leftarrow q$, $P_{N+1, N+1} \leftarrow p$\;
  \For{$r \leftarrow 2$} \KwTo $N${
    $P_{r,r-1} \leftarrow (r-1)/N \cdot q$\;
    $P_{r,r} \leftarrow (N-(r-1))/N \cdot q + (r-1/N \cdot p$\;
    $P_{r,r-1} \leftarrow (N-(r-1))/N \cdot p$\;
  }
  \tcp{Build and initialize Markov chain}
  $X_0 \leftarrow 0$;
  Build Markov chain object\;
  \BlankLine
  Set \pL array, \nT array\, $\epsilon$, initialize \res

  \lForEach{element in \pL}{
    \lForEach{element in \nt} {
      \oT $\leftarrow$ \Floor( ($\epsilon \cdot$ \pL +1) )+1\;
      Initialize \nEO $\leftarrow 0$\;

    \lForEach{$i \leftarrow 1$ \KwTo \nT }{
      Run Markov chain of length \pL\;
      Prepend initial state $X_0$\;
      \If{ $X_0 <$ \oT}{increment \nEO}
    }
    \pEO $\leftarrow$ \nEo$/$\nt\;
    Update results table with \pEo;
   } 
 }
 Report results table\;
\end{algorithm}
\subsection*{Scripts}


\begin{description}

% \item[Geogebra] 

% \link{  .ggb}{GeoGebra applet}

% \item[R] 

% \link{http://www.math.unl.edu/~sdunbar1/   .R}{R script for .}

\begin{lstlisting}[language=R]
library(markovchain)

N <- 51; p <- 1/2; q <- 1 - p;
stateNames <- as.character( 0:N )
## Be careful here, because states numbered from 0,
## but R indexes from 1
transMatrix <- matrix(0, N+1,N+1)
transMatrix[1,1] <- q
transMatrix[1,2] <- p
transMatrix[N+1, N  ] <- q
transMatrix[N+1, N+1] <- p
for (row in 2:N) {
    transMatrix[row, row-1] <- ((row-1)/N)*q
    transMatrix[row,row] <- ((N-(row-1))/N)*q + ((row-1)/N)*p
    transMatrix[row,row+1] <- ((N-(row-1))/N)*p
}

startState <- "0"

altEhernfest <- new("markovchain", transitionMatrix=transMatrix,
                 states=stateNames, name="AltEhernfest")


pathLength <- c(199, 399, 599, 999)
nTrials <- c(200, 400, 800, 1600)

eps <- 0.01

results <- matrix(0, length(pathLength), length(nTrials))

for (pL in 1:length(pathLength)) {
for  (nT in 1:length(nTrials)) {

outlierThreshold <- floor( eps * (pathLength[pL] + 1) ) + 1

nEpsOutlier<- 0

for (i in 1:nTrials[nT]) {
            
    path <- rmarkovchain(n=pathLength[pL],
                         object = altEhernfest,
                         t0 = startState)
    fullPath <- c(startState, path)
    labelFullPath <- as.numeric(fullPath)
    labelStartState <- as.numeric(startState)
    if (!identical(sort(labelFullPath)[outlierThreshold], labelStartState)) {
                nEpsOutlier <- nEpsOutlier + 1
    }
}

probEpsOutlier <- nEpsOutlier/nTrials[nT]

results[pL, nT] <- probEpsOutlier
}
}

results

\end{lstlisting}  

% \item[Octave]

% \link{http://www.math.unl.edu/~sdunbar1/    .m}{Octave script for .}

% \begin{lstlisting}[language=Octave]

% \end{lstlisting}

% \item[Perl] 

% \link{http://www.math.unl.edu/~sdunbar1/    .pl}{Perl PDL script for .}

% \begin{lstlisting}[language=Perl]

% \end{lstlisting}

% \item[SciPy] 

% \link{http://www.math.unl.edu/~sdunbar1/    .py}{Scientific Python script for .}

% \begin{lstlisting}[language=Python]

% \end{lstlisting}

\end{description}



%%% Local Variables:
%%% TeX-master: t
%%% End:


\hr

\visual{Problems to Work}{../../../../CommonInformation/Lessons/solveproblems.png}
\section*{Problems to Work for Understanding}
\renewcommand{\theexerciseseries}{}
\renewcommand{\theexercise}{\arabic{exercise}}

\begin{exercise}
    For a set of real values \( \alpha_0, \dots \alpha_n \), compare the
    definition of ``\( \epsilon \)-outlier'' and ``among-\( \ell \)-smallest''
    to the notions of the order of a list and the rank of elements of a
    set.
\end{exercise}
\begin{solution}
    For a set with no duplicates (or ties), \( \alpha_0 \) is an \(
    \epsilon \)-outlier of the set if and only if the rank (by
    increasing value) of \( \alpha_0 \) in \( \alpha_0, \dots \alpha_n \)
    is less than or equal to \( \epsilon(n+1) \).

    For a set with no duplicates (or ties), \( \alpha_0 \) is among-\(
    \ell \)-smallest if and only its order (increasing) in \( \alpha_0,
    \dots \alpha_n \) is \( \ell + 1 \).

    For a set with duplicates (or ties), the rank is not well-defined
    since there are several methods of ranking the ties.  For example,
    for the R function \texttt{rank()}, the user can specify the
    argument \texttt{ties.method} to determine the result at the
    corresponding indices.  The \texttt{first} method results in a
    permutation with increasing values at each index set of ties, and
    analogously \texttt{last} with decreasing values.  The \texttt{random}
    method puts these in random order whereas the default, \texttt{average},
    replaces them by their mean, and \texttt{max} and \texttt{min}
    replaces them by their maximum and minimum respectively, the latter
    being the typical sports ranking.  See the examples in the R
    documentation for \texttt{rank()}.

    For example, consider the set of \( 11 \) values
    \( \set{\alpha_0 = 3, \alpha_1 = 1, \alpha_2 = 4, \alpha_3 = 1,
      \alpha_4 = 5, \alpha_5 = 9, \alpha_6 = 2, \alpha_7 = 6, \alpha_8
      = 5, \alpha_9 = 3, \alpha_{10} = 5} \) and ask for the value
    with rank \( 1 \).  With the \( \texttt{first} \) method the rank
    \( 1 \) element is \( \alpha_1 = 1 \).  With the
    \( \texttt{last} \) method the rank \( 1 \) element is
    \( \alpha_3 = 1 \).  With the \( \texttt{random} \) method the
    rank \( 1 \) element may be \( \alpha_1 = 1 \) or
    \( alpha_3 = 1 \).  With the \( \texttt{average} \) method the
    rank \( 1 \) element has value \( 1 \).  With the \texttt{max} and \texttt{min}
    methods the rank \( 1 \) elements are \( \alpha_1 = 1 \) or
    \( alpha_3 = 1 \) respectively.

    The
    \( \epsilon \)-outlier
    is well-defined with \( \epsilon = 0.1 \).  Then \(
    \alpha_0 = 1 \) is \emph{not} an \( 0.1 \)-outlier since \( 2 =
    \card{%
    \setof{i}{\alpha_i \le \alpha_0}} \ge \epsilon(n+1) = 1.1 \).  In
  fact,  \( \alpha_0 = 1 \) \emph{is not} a \( 0.2 \)-outlier (the
  cardinality is \( 3 \), greater than \(2.2\))
  or even a \( 0.3 \)-outlier (the cardinality is \( 5 \), greater
  than \( 3.3 \).
\end{solution}

\begin{exercise}
    Given a Markov chain \( X \) with value function \( v:  \mathcal{X} \to
    \Reals \) and stationary distribution \( \pi \), define for each \(
    j, \ell \le n \) a real number \( \rho(j; \ell, n) \) which is the
    probability that for a \( \pi \)-stationary trajectory \( X_1, X_2,
    \dots X_n \) the label \( v(X_j) \) is among-\( \ell \)-smallest in \( v(X_1),
    v(X_2), \dots v(X_n) \).  Show that \( \rho(j,\ell, n) \) is
    well-defined.  
\end{exercise}
\begin{solution}
  The probability of a \( \pi \)-stationary trajectory \( X_1, X_2,
    \dots X_n \) can be calculated as usual.  Then the probability of
    the value set \( v(X_1), v(X_2), \dots v(X_n) \) can be
    calculated.  The label \( v(X_j) \) is among-\( \ell \)-smallest in \( v(X_1),
    v(X_2), \dots v(X_n) \) is a sub-event of the event \( v(X_1),
    v(X_2), \dots v(X_n) \).  Then \( \rho(j; \ell, n) \) which is the
    probability that for a \( \pi \)-stationary trajectory \( X_1, X_2,
    \dots X_n \) the label \( v(X_j) \) is among-\( \ell \)-smallest in \( v(X_1),
    v(X_2), \dots v(X_n) \) is the probability of this sub-event.
  \end{solution}
  
\begin{exercise}
    Use the script to determine the proportion of trials for which the
    state \( 0 \) is an \( 0.01 \)-outlier for the alternative Ehrenfest
    urn model with \( N = 7, 21, 51 \) balls and path length of the
    chain \( 199, 399, 599, 999 \) steps when executing \( 200, 400, 800
    \) and \( 1600 \) trials.  Display the results in three charts, one
    for each \( N \).
\end{exercise}
\begin{solution}

\begin{lstlisting}[language=R]
library(markovchain)

N <- 7; p <- 1/2; q <- 1 - p;
stateNames <- as.character( 0:N )
## Be careful here, because states numbered from 0,
## but R indexes from 1
transMatrix <- matrix(0, N+1,N+1)
transMatrix[1,1] <- q
transMatrix[1,2] <- p
transMatrix[N+1, N  ] <- q
transMatrix[N+1, N+1] <- p
for (row in 2:N) {
    transMatrix[row, row-1] <- ((row-1)/N)*q
    transMatrix[row,row] <- ((N-(row-1))/N)*q + ((row-1)/N)*p
    transMatrix[row,row+1] <- ((N-(row-1))/N)*p
}

startState <- "0"

ehrenfest2 <- new("markovchain", transitionMatrix=transMatrix,
                 states=stateNames, name="Ehrenfest2")


pathLength <- c(199, 399, 599, 999)
nTrials <- c(200, 400, 800, 1600)

eps <- 0.01

results <- matrix(0, length(pathLength), length(nTrials))

for (pL in 1:length(pathLength)) {
for  (nT in 1:length(nTrials)) {

outlierThreshold <- floor( eps * (pathLength[pL] + 1) ) + 1

nEpsOutlier<- 0

for (i in 1:nTrials[nT]) {
            
    path <- rmarkovchain(n=pathLength[pL],
                         object = ehrenfest2,
                         t0 = startState)
    fullPath <- c(startState, path)
    labelFullPath <- as.numeric(fullPath)
    labelStartState <- as.numeric(startState)
    if (!identical(sort(labelFullPath)[outlierThreshold], labelStartState)) {
                nEpsOutlier <- nEpsOutlier + 1
    }
}

probEpsOutlier <- nEpsOutlier/nTrials[nT]

results[pL, nT] <- probEpsOutlier
}
}

results
\end{lstlisting}

\begin{description}
\item[\( N= 7 \)\
\begin{verbatim*}
        [,1]   [,2]    [,3]     [,4]
[1,] 0.430 0.3975 0.38875 0.407500
[2,] 0.475 0.4800 0.48125 0.488125
[3,] 0.555 0.5675 0.52500 0.530000
[4,] 0.545 0.5900 0.59750 0.577500
\end{verbatim*}
\item[\( N = 21 \)]
\begin{verbatim*}
      [,1]   [,2]    [,3]     [,4]
[1,] 0.730 0.7250 0.75250 0.740625
[2,] 0.875 0.9200 0.92625 0.920000
[3,] 0.985 0.9825 0.98625 0.975000
[4,] 1.000 1.0000 1.00000 0.999375
\end{verbatim*}
\item[\( N = 51 \)]
\begin{verbatim*}
      [,1]   [,2]    [,3]     [,4]
[1,] 0.775 0.7425 0.73500 0.734375
[2,] 0.930 0.9275 0.94125 0.935625
[3,] 0.980 0.9850 0.98375 0.979375
[4,] 0.995 0.9975 0.99875 1.000000
\end{verbatim*}
\end{description}
\end{solution}

These experiments show that
the proportion of trials for which the
    state \( 0 \) is an \( 0.01 \)-outlier for the alternative Ehrenfest
    urn model is reasonably constant with respect to the length of the
    chain, but slowly increases with the number of trials.  The
    experiments also show that
the proportion of trials for which the
    state \( 0 \) is an \( 0.01 \)-outlier for the alternative Ehrenfest
    urn model increases with the number of values allowed by the value
    function, or equivalently the sensitivity of the value function.  

\begin{exercise}
Consider the bathroom occupancy example in the section on Stationary
Distributions.  The Markov chain considered there is irreducible and
aperiodic and because it is formulated as a random walk on a weighted
graph it is reversible.  Let the states be valued as ``OOO''$=1$,
``OOV''$=2$, ``OFO''$=3$, ``OCO''$=4$, ``VO''$=5$, ``VV''$=6$, so the greater
value is a more desirable state.  For $\epsilon=0.1$ and a path
length of $1000$ steps starting from ``OOO'', determine if the least
desirable state ``OOO'' is unusual with respect to this value function
in the sense of being an $\epsilon$-outlier.

The transition probability matrix is
\[
    P= \bordermatrix{ & OCO & OFO & OOO & OOV & VO & VV \cr
    OCO & 1/2 & 0 & 0 & 0 & 0 & 1/2 \cr
    OFO & 0 & 1/2 & 0 & 0 & 1/4 & 1/4 \cr
    OOO & 0 & 0 & 1/2 & 0 & 1/2 & 0 \cr
    OOV & 0 & 0 & 0 & 1/2 & 1/2 & 0 \cr
    VO & 1/6 & 1/6 & 1/6 & 0 & 1/2 & 0 \cr
    VV & 1/6 & 1/6 & 0 & 1/6 & 0 & 1/2 \cr
    }.
\]
\end{exercise}

\begin{solution}
  Using this R script, the answer is \emph{yes}.  This says that starting from
  ``OOO'', the probability of seeing such an unusual state with respect to
  this value function is less than $0.1$.  While this is not a
  terribly small probability, in the context of the bathroom example,
  it is somewhat reassuring!

  \begin{lstlisting}[language=R]
library("markovchain")

rows = 6

stateNames <- c("OCO","OFO","OOO", "OOV", "VO", "VV")
P <- matrix(c(1/2,   0,  0,  0,   0, 1/2,
                0, 1/2,  0,  0, 1/4, 1/4,
              0,   0, 1/2, 0, 1/2,   0,
              0,   0, 0,   1/2, 1/2, 0,
              1/6, 1/6, 1/6, 0, 1/2, 0,
              1/6, 1/6,  0, 1/6, 0, 1/2), 
             nrow=rows, byrow=TRUE)
rownames(P) <- stateNames
colnames(P) <- stateNames

startState  <- "OOO"

bathroom  <- new("markovchain", transitionMatrix=P,
                 states=stateNames, name="Bathroom_Occupancy")

pathLength <- 1000

eps <- 0.1
outlierThreshold <- floor( eps * (pathLength + 1) ) + 1

v <- function (x) {
## value the states, greater value is more desirable state
    switch(x, "OOO"=1, "OOV"=2, "OFO"=3, "OCO"=4, "VO"=5, "VV"=6)
}    

bathroomHistory  <- rmarkovchain(n=pathLength, object=bathroom,
                                 t0=startState)

values <- rep(0, pathLength)

for (i in 1:pathLength) {
    values[i] <- v(bathroomHistory[i])
}    

if (!identical(sort(values)[outlierThreshold], v(startState))) {
       print(cat( startState, "IS an", eps, "outlier for the bathroom states\n"))
   } else {
       print(cat( startState, "is NOT an", eps, "outlier for the bathroom states\n"))
}
\end{lstlisting}

This should not be too surprising, since the probability of ``OOO''
occurring in the stationary distribution is $1/12$, so in $n=1000$  steps
in the stationary distribution, one could expect to see ``OOO'' about
$80$ times, near the threshold of $\epsilon \cdot n = 100$.  The
second largest eigenvalue of $P$ is $1/2$, so expect the Markov chain to have a
total variation distance from stationary in about $0.01$ in about $7$
steps or so.
\end{solution}

% Examples of outliers for reversible chains.
%         \begin{enumerate}
%             \item
%                 bathroom example in Stationary Distributions (reversible?)
%             \item
%                 Random walk on a circle
%             \item
%                 Card shuffling on 3 cards
%             \item
%                 Gibbs sampler for bivariate normal (reversible?)
%             \item
%                 Use at least one of these for exercises
%         \end{enumerate}

\begin{exercise}
    Pennsylvania is divided into roughly \( 9{,}000 \) census block groups.
    Define a division of these blocks into \( 18 \) districts to be a
    valid districting of Pennsylvania if districts are equal in
    population. For purposes of the problem, assume
    that census block groups are approximately equal in
    size. (This is not a valid valid assumption, census block groups
    can have widely varying populations between rural and urban
    areas.) Also assume the only constraint on valid districting is that
    the districts have roughly equal population. (This is not the only
    constraint, usually district are also required to be connected and
    to be compact, that is, have no extremely contorted shapes.)  Justify the claim
    that there would be \( 10^{10000} \) or so valid districtings
    under this simple assumption.
\end{exercise}

\begin{solution}
    Assuming that census block are approximately equal in size, and the
    only constraint on valid districting is that the districts have
    roughly equal population, then each district would have \( 500 \)
    blocks.  The way to allocate \( 9000 \) blocks in \( 18 \) groups of
    \( 500 \) each is the multinomial
    \[
        N = \frac{9000!}{(500!)^{18}}
    \] With Stirling's approximation, this is approximately
    \begin{align*}
        N &\approx \frac{\sqrt{2\pi} (9000)^{9000} \sqrt{9000} \EulerE^{-9000}}
        {(\sqrt{2\pi} (500)^{500} \sqrt{500} \EulerE^{-500})^{18}} \\
        &= (2\pi)^{-17/2} \left( \frac{9000}{500} \right)^{9000} \left(
        \frac{\sqrt{9000}}{500^{9}} \right) \\
        &= (2\pi)^{-17/2} 18^{9000} \left( \frac{\sqrt{9000}}{500^{9}}
        \right) \\
    \end{align*}
    Since all that is needed is the exponent, evaluate the base-\( 10 \)
    logarithm of the last expression.  That is \( \log_{10} N \approx 11
    {,}268 \).  So the assertion that there are \( 10^{10000} \) or so
    equal population districtings is a justified lower estimate.

    Just for comparison, scientists estimate there are about \( 10^{80} \)
    \emph{atoms} in the universe.
  \end{solution}

  \begin{exercise}
  Cite a reference to a theorem from classical probability for the probability that an \( n
\)-step \( 1 \)-dimensional random walk from \( X_0 \) takes a first
step to the right and does not return to the origin.
  \end{exercise}
  \begin{solution}
    This is Corollary 10.4, page 63 of \title{Heads or Tails: An
      Introduction to Limit Theorems in Probability}, Emmanuel
    Lesigne, AMS Student Mathematical Library, Volume 28, 2005.   This
    is also formulas (3.5) and (2.3), on pages 77 and 75 respectively
    in W. Feller's \title{An Introduction to Probability Theory and Its
    Applications, Volume I.}, John Wiley and Sons, 1973.
\end{solution}

  \begin{exercise}
    Verify that
    \[
    \frac{1}{2^{n+1}}\binom{n}{n/2} \asympt \frac{1}{\sqrt{2 \pi n}}.
\]
\end{exercise}
\begin{solution}
  Using Stirling's Approximation is one way, slightly tedious,  to
  show this.

  Another way is to start from the inequality
  \[
    \frac{4^n}{\sqrt{(2n+1)(\PI/2)}}\left(1-\frac{1}{2n} \right) < {\binom
    {2n}{n}} < \frac{4^n}{\sqrt{(2n+1)(\PI/2)}} \left(1 + \frac{1}{2n}
    \right).
\]
 established in the subsection \emph{Wallis' Formula and the Central
 Binomial Coefficient} of the section \emph{Wallis Formula} in the
chapter on Stirling's Formula.  Rewrite \( n = 2n \) and divide
through by \( 2^{n + 1} = 2 \cdot 4^n \) to obtain
\[
  \frac{1}{\sqrt{2\pi(n+1)}} \left( 1 - \frac{1}{n} \right) <
  \frac{1}{2^{n+1}} \binm{n}{n/2} <
  \frac{1}{\sqrt{2\pi(n+1)}} \left( 1 + \frac{1}{n} \right).
\]
Then it quickly follows 
    \[
    \frac{1}{2^{n+1}}\binom{n}{n/2} \asympt \frac{1}{\sqrt{2 \pi n}}.
\]
\end{solution}
% \begin{exercise}
%   \begin{enumerate}[label=(\alpha*)]
%   \item
%   \end{enumerate}
% \end{exercise}
% \begin{solution}
%   \begin{enumerate}[label=(\alpha*)]
%   \item
%   \end{enumerate}
% \end{solution}

\hr

\visual{Books}{../../../../CommonInformation/Lessons/books.png}
\section*{Reading Suggestion:}

\bibliography{../../../../CommonInformation/bibliography}

%   \begin{enumerate}
%     \item
%     \item
%     \item
%   \end{enumerate}

\hr

\visual{Links}{../../../../CommonInformation/Lessons/chainlink.png}
\section*{Outside Readings and Links:}
\begin{enumerate}
    \item
    \item
    \item
    \item
\end{enumerate}

\section*{\solutionsname} \loadSolutions

\hr

\mydisclaim \myfooter

Last modified:  \flastmod

\end{document}

%%% Local Variables:
%%% mode: latex
%%% TeX-master: t
%%% End:

%%% Local Variables:
%%% TeX-master: t
%%% End:
